\documentclass[12pt,a4paper,oneside,draft]{report}

\begin{document}

\title{SimpleDB: Course Project of Compiler Theory}
\date{\today}
\author{Sun Jiacheng}
\maketitle

\chapter{Information}
\section{Team Information}
\begin{tabular}{l c r}
Sun Jiacheng & 12330285 & 291624707@qq.com\\
Kuang Yuanyuan & 12330153 & 596755905@qq.com\\
Qiu Zhilin & 12330268 & 847300960@qq.com\\
Sun Dongliang & 12330284 & 1255993541@qq.com\\
Wang Kaibin & 12330305 & wkbpluto@qq.com\\
\end{tabular}

\section{Tarball Structure}
\begin{verbatim}
/doc -- documents of project
/src -- source code
\end{verbatim}

\section{Compile}
\begin{verbatim}
cd ./src
make -- compile SimpleDB
make test -- compile and run sample test
make tar -- tar the project.
\end{verbatim}

\chapter{Project}

\section{Project Website}
https://github.com/thomasking0529/SimpleDB\\

\section{Code Style}
K\&R(CDT's default style)\\ 

\section{Common}

\subsection{Lexer}
\begin{verbatim}
enum TokenType:
    KEYWORD -- actions(create, insert, delete, select),
          primitives("table" only), case insensitive,
          "where", properties of table(default, primary
          key), case insensitive. 
          
    ID -- id (identifier) is a sequence of digits, underline
          and letters. All identifiers should start with a
          letter or an underline. The maximum length of an
          identifier is 64.
          
    NUM -- num (number) is a sequence of digits. (of 32-bits)
    
    OP -- Arithmetical operators: +, -, *, /, unary -, unary +
          Relational operators: <, >, <>, ==, >=, <=
          Logical operators: &&, ||, !
          Assignment operator: =.
          Basic punctuation("(", ")", ",")

struct Token:
    TokeType type -- token type.
    std::string value -- token value, store the original string
          of token.

\end{verbatim}

\subsection{Parser}
\begin{verbatim}
enum Action:
    CREATE -- create table
    INSERT -- insert one row to table
    DELETE -- delete row(s) from table
    SELECT -- query row(s) from table
    INVALID -- if unknown keywords detected

enum Op:
    PLUS, // +, both unary and binary
    MINUS, // -, both unary and binary
    MULTIPLY, // *
    DIVIDE, // /
    LT, // <
    GT, // >
    NE, // <>
    E, // ==
    GTE, // >=
    LTE, // <=
    AND, // &&
    OR, // ||
    NOT, // !
    EQ, // =
    LB, // (
    RB, // )
    COMMA, // ,

struct Condition:
    std::string lop -- left operand
    std::string rop -- right operand
    Op op -- operator

enum PropType:
    INT -- int(only)

struct Property:
    std::string id -- property id
    PropType type -- property type    
    std::string default_value -- default value
    operator== -- operator overloading
    
struct Statement:
    Action act -- action.
    std::string table -- table to operate on.
    std::list<std::string> prop_list -- property to return or
        add, for create and select
    std::list<Condition> conds -- where clauses, for create,
        select and delete
\end{verbatim}

\subsection{Core}
\begin{verbatim}
struct Table:
    std::string id -- table name
    std::set<Property> -- properties of table
    std::list<std::string> -- values of properties, in 1-D array
    void insert(const std::list<std::string>& record) -- insert
     
\end{verbatim}

\section{Design}
How you implement.
Time and space complexity.
\subsection{Lexer}
Lexer reads a single string of statement and extracts tokens from it.
\subsubsection{APIs}
\begin{verbatim}
Constructor:
    Initialization.
std::list<Token> GetTokens(const std::string& statement):
    Get a string of statement and return a list of tokens extracted.
    Do token validation.
\end{verbatim}

\subsection{Parser}
Parser get a list of tokens and return a single statement.
\subsubsection{APIs}
\begin{verbatim}
Constructor:
    Initialization.
Statement Parse(std::list<Token> token_list)
    Parse token list to statement.
    Invoke separated parse function.
    Do grammar validation.
\end{verbatim}

\subsection{Core}
Core part of SimpleDB is supposed to manage the "true" database in memory.\\
\subsubsection{APIs}
\begin{verbatim}
Constructor:
    Initialization.
void Execute(std::string)
    Execute a single statement.
    Invoke private member function.
    Check consistency. 
\end{verbatim}

\section{Distribution}
\subsection{Lexer implementation}
\subsection{parseCreate and parseInsert}
\subsection{parseDelete and parseSelect}
\subsection{parseWhere}
\subsection{Core}

\section{Test}
● A test sample contains:
– a single statement or several statements to test the
program
– what is this test sample for
– what's the expected output
– does the program work correctly
● TAs will have their own test samples to test
your program.
● Test document: does the interpreter work?
– Test samples and screenshots

\end{document}